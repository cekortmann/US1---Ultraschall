\section{Diskussion}
\label{sec:Diskussion}

Bei der Messung der Schallgeschwindigkeit kann der experimentelle Wert $v = (2718\pm 25) \, \unit{\m\per\s}$ mit dem theoretischen Wert 
$2700 \,\unit{\m\per\s}$ verglichen werden. Die experimentell bestimmte Schallgeschwindigkeit stimmt im Rahmen der Fehlerabweichung mit 
dem theoretischen Wert überein. Die Abweichung zwischen beiden Werten beträgt $0.6\,\%$. 

\subsection{Untersuchung des Acrylblocks mit einem A-Scan}
Sowohl die Werte des A-Scans als auch die Ergebnisse des B-Scans lassen sich mit den mit der Schieblehre ausgemessenen Werten %sowie untereinander 
vergleichen.

Die Abweichungen des A-Scans zu den mit der Schieblehre gemessenen Werten sind in \autoref{tab:AAbweichung} dargestellt. 

\begin{table}
    \centering 
    \caption{Abweichungen zwischen dem A-Scan und den mit der Schieblehre bestimmten Werten.}
\begin{tabular}{c c c c}
    \toprule
    & \multicolumn{3}{c}{Prozentuale Abweichung der Messungen des A-Scans.} \\
    Bohrung & von unten & von oben & Durchmesser\\
    \midrule
    1&0.5&1.6&35.3 \\
    2&0.3&2.3&35.3 \\
     3&1.5&0.5&6.7 \\
     4&0.9&0.4&8.0 \\
    5&1.7&0.4&10.0 \\
    6&1.0&1.1&16.7 \\
     7&0.4&1.7&7.7 \\
     8&0.4&2.6&7.7 \\
     9&0.5&2.9&7.7 \\
    10&5.9&--&-- \\
    11&2.6&0.6&4.0 \\
    \bottomrule
\end{tabular}
\label{tab:AAbweichung}
\end{table}
Dass die Messwerte nicht genau übereinstimmen, kann an ungenauen Messungen mit der Schieblehre sowie der Ungenauigkeit der Dicke der Anpassungsschicht 
liegen, die sich auf die Messergebnisse des A-Scans fortpflanzt. Die Abweichungen sind allgemein bei kleineren Messwerten größer, da bei einer kleinen 
Größenordnung der Messfehler mehr ins Gewicht fällt. Allgemein lässt sich sagen, dass die Messung mittels Ultraschall/A-Scan vermutlich genauere Ergebnisse 
liefert.

\subsection{Untersuchung des Acrylblocks mit einem B-Scan.}
\begin{table}
    \centering 
    \caption{Abweichungen zwischen dem B-Scan und den mit der Schieblehre bestimmten Werten.}
\begin{tabular}{c c c c}
    \toprule
    & \multicolumn{3}{c}{Prozentuale Abweichung der Messungen des B-Scans.} \\
    Bohrung & von unten & von oben & Durchmesser\\
    \midrule
    1&1.2&4.2&88.2 \\
    2&0.7&4&64.7 \\
     3&4.5&1.2&23.3 \\
     4&3.7&1.5&68.0 \\
    5&6.3&1.8&67.5 \\
    6&2.8&2.6&50.0 \\
     7&2.1&3.4&50 \\
     8&1.1&0.9&33.3 \\
     9&1.1&7.3&40 \\
    10&--&1.5&-- \\
    11&5.2&0.5&13.8 \\
    \bottomrule
\end{tabular}
\label{tab:BAbweichung}
\end{table}
In \autoref{tab:BAbweichung} sind die Abweichungen vom B-Scan zu den Messungen der Schieblehre aufgetragen. Bei der Bestimmung des Durchmessers sind Abweichungen von $>50\%$ keine Seltenheit.
Die Abweichung des B-Scans weichen stark von den Werten, die mit der Schieblehre gemessen wurden ab. Grund hierfür könnte unter anderem die ungenaue Messung sein. 
Wie in \autoref{fig:bunten} und \autoref{fig:foben} zu sehen, nehmen die Löcher des Acrylquaders einen großen Bereich ein ohne deutliche Abgrenzungen zu haben.
Aufgrund der undeutlichen Grenzen war es also häufig nicht möglich einen genauen Punkt zu finden wo das Loch aufhört. Die Schieblehre scheint hier ein zuverlässigeres Mittel 
als der B-Scan zu sein.
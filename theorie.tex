\section{Theorie}
\label{sec:Theorie}


Der Mensch kann Töne in einem Frequenzbereich von etwa $16-20000\, \unit{\hertz}$ wahrnehmen. Der Bereich über der Hörschwelle wird 
Ultraschall genannt.

\subsection{Die Physik der Schallwelle}

Der Schall ist eine longitudinale Well dessen Ausbreitung über die Gleichung
\begin{equation*}
    p(x,t)=p_0+v_0Zcos(\omega t-kx)
\end{equation*}
beschrieben werden kann. Die Größe $Z= c*\rho$ beschreibt hierbei die akustische Impedanz, welche durch die Dichte und die Schallgeschwindigkeit in dem
Material beschrieben wird.
Wie dem obenstehenden zu entnehmen ist, bereitet sich Schall in verschiedenen Medien mit unterschiedlichen Phasengeschwindigkeiten aus.
In Flüssigkeiten und Gasen breitet sich der Schall stets longitudinal.
Die Schallgeschwindigkeit kann so über 
\begin{equation*}
    c_{\text{Fl}}=\sqrt{\frac{1}{\kappa \cdot \rho}}
\end{equation*}
,wobei $\kappa$ die Kompressibilität des Mediums beschreibt, errechnet werden.
Bei einem Festkörper hingegen verläuft die Ausbreitung nicht nur longitudinal, sondern kann auch Transversal vonstatten gehen.
Die Formel
\begin{equation*}
    c_{\text{Fe}}=\sqrt{\frac{E}{\rho}}
\end{equation*}
beschriebt dient zur Berechnung der Schallgeschwindigkeit in festen Materialien.
Wie bereits oben erwähnt, kann die Ausbreitungsrichtung in zwei Richtungen ablaufen, folglich ist auch deren Geschwindigkeit eine andere
Schall verliert bei der Ausbreitung einen Teil der Energie durch Absorption.
Die Abnahme der Intensität kann durch 
\begin{equation*}
    I(x)=I_0\cdot e^{\alpha x}
\end{equation*}
mit dem Absorptionskoeffizienten $\alpha$ beschrieben werden.
Weiter wird Schall beim Auftreffen auf eine Grenzfläche teilweise reflektiert.
Das Verhältnis von reflektierter zur eintreffenden Schallintensität wird Reflexionskoeffizient $R$ genannt
\begin{equation*}
    R=\left(\frac{Z_1-Z_2}{Z_1+Z_2}\right)^2\, .
\end{equation*}
$Z_1$ steht hierbei für die akustische Impedanz des ersten Materials, analog für $Z_2$.
Der transmittierte Anteil lässt sich über $T=1-R$ bestimmen.

\subsection{Erzeugung von Ultraschall}
\label{sec:Erzeugung}
Die Erzeugung von Ultraschallwellen wird in diesem Versuch über den reziproken piezo-elektrischen Effekt erzeugt.
Hierzu wird ein piezoelektrischer Kristall einem elektrischen Wechselfeld ausgesetzt.
Der Kristall kann zum Schwingen angeregt werden, wenn einer seiner polaren Achsen auf in die Richtung des elektrischen Feldes liegt.
Durch die Kraft auf die im Kristall befindlichen Ladungsträger verformt dieser sich leicht, sodass die Verformung mikroskopische Dipole erzeugt.
Durch die oszillierende Richtung des Feldes beginnt auch die Verformung zu Schwingen.
Hierbei wird Ultraschall emittiert. 
Als Piezokristall dient häufig Quarz, welches aufgrund seiner gleichbleibenden physikalischen Eigenschaften kaum durch äußere Umstände beeinflusst wird, jedoch
nur einen relativ schwachen piezoelektrischen Effekt hat.

\subsection{Erklärung des Impuls-Echo-erfahrens}
\label{sec:Impuls-Echo}
Als Grundlage der anstehenden Messungen dient das sgn. Impuls-Echo-Verfahren.
Hierbei dient der Ultraschallsender auch zeitgleich als Empfänger, welcher die von der Grenzfläche reflektierten Schallwellen aufnimmt, nachdem er diese ausgesendet hat. 
Trifft der Ultraschall also auf Unebenheiten, Fremdkörper oder sonstige Änderungen der Materialeigenschaften, kann das verwendete Verfahren Aufschluss über dessen Größe geben. 
Bei bekannter Schallgeschwindigkeit kann die Lage der Fehlstelle aus der Laufzeit $t$ bestimmt werden
\begin{equation*}
    s=\frac{1}{2}ct\, .
\end{equation*}

Die benötigten Messwerte werden über verschiedene Verfahren aufgenommen.
Der A-Scan (Amplituden Scan) ist ein eindimensionales Verfahren bei dem die Echoamplituden gegen die Laufzeit aufgetragen wird. Beim B-Scan (Brightness Scan) wird durch Bewegen der Sonde ien zweidimensionales aufgenommen,
auf welchem die Echoamplituden in Helligkeitsstufen dargestellt werden, sodass ein zweidimensionales Schnittbild erzeugt wird.

\section{Durchführung}
\label{sec:Durchfuehrung}

Zuerst wird der Acrylblock mit einer Schieblehre ausgemessen. Neben der Höhe, Tiefe und 
Breite des Acrylblocks werden außerdem die Positionen und Durchmesser der Bohrungen bestimmt.

Danach werden für sieben Bohrungen die Laufzeiten mittels des Impuls-Echo-Verfahrens gemessen, um 
damit die Schallgeschwindigkeit und die Dicke der Anpassungsschicht zu bestimmen.

\subsection{Untersuchung des Acrylblocks mit dem A-Scan}
Nun werden mittels eines A-Scans die Größe und Positionen aller Bohrungen bestimmt. Dazu wird die 
2-MHz-Sonde verwendet. Als Kontaktmittel zwischen dem Acrylblock und der Sonde dient destilliertes Wasser.
Danach wird der Acrylblock umgedreht und erneut ein A-Scan von allen Bohrungen aufgenommen.

\subsection{Untersuchung des Auflösungsvermögens}
Um das Auflösungsvermögen zu bestimmen, werden die beiden benachbarten Bohrungen 1 und 2 mit einer 1-MHz-Sonde 
und einer 2-MHz-Sonde untersucht und die erhaltenen Graphen dann miteinander verglichen.

\subsection{Untersuchung des Acrylblocks mit dem B-Scan}
Erneut wird der Acrylblock von beiden Seiten mit einer 2-MHz-Sonde untersucht. Dabei wird ein B-Scan beider 
Seiten erstellt, um die Abmessungen der Bohrungen zu bestimmen.

\subsection{Untersuchung eines Brustmodells mit einem B-Scan}
Zuerst wird dei ungefähre Lage der beiden Tumore in dme Brustmodell ertastet. Nun wird mit der 2-MHz-Sonde 
auf einer gedachten Verbindungslinie zwischen den ertasteten Stellen ein B-Scan aufgezeichnet, welcher 
sich qualitativ erklären lässt. Hieraus ist auch die Art des Tumors bestimmbar.
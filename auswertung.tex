\section{Auswertung}
\label{sec:Auswertung}

\subsection{Fehlerrechnung}
\label{sec:Fehlerrechnung}
Für die Fehlerrechnung werden folgende Formeln aus der Vorlesung verwendet.
für den Mittelwert gilt
\begin{equation}
    \overline{x}=\frac{1}{N}\sum_{i=1}^N x_i ß\; \;\text{mit der Anzahl N und den Messwerten x} 
    \label{eqn:Mittelwert}
\end{equation}
Der Fehler für den Mittelwert lässt sich gemäß
\begin{equation}
    \increment \overline{x}=\frac{1}{\sqrt{N}}\sqrt{\frac{1}{N-1}\sum_{i=1}^N(x_i-\overline{x})^2}
    \label{eqn:FehlerMittelwert}
\end{equation}
berechnen.
Wenn im weiteren Verlauf der Berechnung mit der fehlerhaften Größe gerechnet wird, kann der Fehler der folgenden Größe
mittels Gaußscher Fehlerfortpflanzung berechnet werden. Die Formel hierfür ist
\begin{equation}
    \increment f= \sqrt{\sum_{i=1}^N\left(\frac{\partial f}{\partial x_i}\right)^2\cdot(\increment x_i)^2}.
    \label{eqn:GaussMittelwert}
\end{equation}
Die Abmessungen der einzelnen Bohrungen befinden sich in \autoref{tab:block}. Die Nummerierung der Bohrungen befinden sich in \autoref{fig:acryl}.
\begin{table}
    \centering 
    \caption{Abmessungen der einzelnen Bohrungen}
\begin{tabular}{c c c c}
    \toprule
    Bohrung & Abstand von unten in \unit{\mm} & Abstand von oben \unit{\mm} & Durchmesser d in \unit{\mm}\\
    \midrule
    1&59.0&19.2&1.7 \\
     2&60.7&17.5&1.7 \\
     3&13.2&60.6&6.0 \\
     4&21.7&53.2&5.0 \\
     5&30.0&45.6&4.0 \\
     6&38.6&38.0&3.0 \\
     7&46.6&29.8&3.0 \\
     8&54.7&22.8&3.0 \\
     9&62.7&13.7&3.0 \\
     10&70.7&6.8&3.0 \\
    11&15.3&54.4&9.9 \\
    \bottomrule
    \end{tabular}
    \label{tab:block}
\end{table}

Um die Schallgeschwindigkeit und die Dicke der Anpassungsschicht zu bestimmen, werden für sieben Bohrungen die Laufzeiten bestimmt. Diese befinden 
sich in \autoref{tab:geschw}.

\begin{table}
    \centering 
    \caption{Messwerte der Laufzeit der einzelnen Bohrungen}
\begin{tabular}{c c}
    \toprule
    Bohrung & Laufzeit $t$ in \unit{\micro\sec}\\
    \midrule
    3&45.4 \\
      4&39.9 \\
      5&34.4 \\
      6&29.0 \\
      7&23.0 \\
      8&17.1 \\
      9&11.1 \\
      \bottomrule
    \end{tabular}
    \label{tab:geschw}
\end{table}

Nun werden die Abstände der Bohrungen von der oberen Fläche gegen die Laufzeiten aufgetragen und eine Ausgleichsgerade gebildet. Diese befinden sich in 
\autoref{fig:plotv}.

